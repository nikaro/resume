\documentclass[10pt,a4paper,roman]{moderncv}
\moderncvtheme[blue]{classic}
\usepackage[utf8]{inputenc}
\usepackage[scale=0.80]{geometry} % La taille pris par le contenu, ici on a 15% de marges.
\nopagenumbers{} % Permet de masquer les numéros de page
 
\title{Ingénieur Systèmes Linux}
\name{Nicolas}{Karolak}
\address{62 rue Paul Gauguin}{77550 Moissy-Cramayel}
\mobile{06 51 42 21 38}
\email{nicolas@karolak.fr}
\homepage{blog.karolak.fr}
\social[linkedin]{nicolas-karolak}
\social[github]{nikaro}
% \extrainfo{\\}
% \extrainfo{31 ans\\Permis B}
 
\begin{document}
\maketitle
 
\section{Expériences}
\cventry{Mai 2020\\à Aujourd'hui}{Administrateur Systèmes et Réseaux}{Univ. Gustave Eiffel}{Champs-sur-Marne}{France}{
    \begin{itemize}
        \item Déploiement de services (WireGuard, pfSense, etc.).
        \item Écriture d'un démon de logging pour WireGuard.
    \end{itemize}
}
\cventry{Oct. 2012\\à Aujourd'hui}{Enseignant vacataire en Licence Pro}{}{}{}{
    \begin{itemize}
        \item Scripting (Python)
        \item Chiffrement et certificats
        \item VPN
    \end{itemize}
}
\cventry{Sep. 2017\\à Avril 2020}{Ingénieur Systèmes Linux}{UbiCast}{Ivry-sur-Seine}{France}{
    \begin{itemize}
        \item Déploiement d'infrastructure avec Terraform, Packer et Ansible.
        \item Portage des scripts de déploiement (Python/Bash) vers Ansible.
        \item Mise en haute-disponibilité du produit (PgSQL, HAProxy, etc.).
    \end{itemize}
}
\cventry{Sep. 2016\\à Août 2017}{Responsable informatique}{Lycée Technique Privé Saint-Nicolas}{Paris}{France}{
    \begin{itemize}
        \item Développement d'un gestionnaire de mots de passe (Django).
        \item Migration serveur de messagerie interne (Postfix, Dovecot).
        \item Mise en place Nextcloud pour les professeurs et étudiants.
    \end{itemize}
}
\cventry{Sep. 2015\\à Sep. 2016}{Administrateur Systèmes et Réseaux}{IUT de Sénart}{Lieusaint}{France}{
    \begin{itemize}
        \item Administration système (Debian/Ubuntu).
        \item Développement (PHP/Bash/Python).
        \item Exploitation de pars (Ubuntu/Windows).
    \end{itemize}
}
\cventry{Déc. 2014\\à Sep. 2015}{Développeur Web}{Indépendant}{}{France}{
    \begin{itemize}
        \item Développement de sites internet en PHP et Python/Django.
    \end{itemize}
}
\cventry{Août 2011\\à Déc. 2014}{Administrateur Systèmes et Réseaux}{Lycée Technique Privé Saint-Nicolas}{Paris}{France}{
    \begin{itemize}
        \item Administration serveurs GNU/Linux et Windows.
        \item Virtualisation, VPN, Active Directory, etc.
    \end{itemize}
}
 
\section{Formations}
 
\cventry{2011 -- 2012}{Licence Pro Administration Systèmes et Réseaux}{IUT de Sénart}{Lieusaint}{France}{}{}
\cventry{2009 -- 2011}{BTS Informatique de Gestion}{Institut Medicis Alternance}{Paris}{France}{}{}
\cventry{2007 -- 2009}{Bac Pro Micro-informatique et Réseaux}{Lycée Jacques Prévert}{Combs-la-Ville}{France}{}{}
 
\section{Compétences}

\subsection{Informatique}

\cvline{Langages}{Bash, Python, Go, HTML/CSS, PHP}
\cvline{OS}{Debian, CentOS, ArchLinux}
\cvline{Outils}{Git}

\subsection{Langues}

\cvline{Anglais}{Professionnel}
 
\section{Centres d'intérêt}
 
\cvline{Loisirs}{Guitare en autodidacte depuis 4 ans.}
\cvline{}{Pratique du Viet Vo Dao, durant 7 ans.}
 
\end{document}
